\capitulo{4}{Técnicas y herramientas}

Esta parte de la memoria tiene como objetivo presentar las técnicas metodológicas y las herramientas de desarrollo que se han utilizado para llevar a cabo el proyecto. Si se han estudiado diferentes alternativas de metodologías, herramientas, bibliotecas se puede hacer un resumen de los aspectos más destacados de cada alternativa, incluyendo comparativas entre las distintas opciones y una justificación de las elecciones realizadas. 
No se pretende que este apartado se convierta en un capítulo de un libro dedicado a cada una de las alternativas, sino comentar los aspectos más destacados de cada opción, con un repaso somero a los fundamentos esenciales y referencias bibliográficas para que el lector pueda ampliar su conocimiento sobre el tema.


\section{Stream Processing Framework}

\subsection{Apache Kafka}

Apache Kafka es un sistema de intermediación de mensajes que respone al patrón "Publish/Subscribe Messaging" que se utiliza para la comunicación entre aplicaciones. Entre sus principales características se encuentran que es un sistema escalable y persistente, con gran tolerancia a fallos y gran velocidad tanto de escritura cómo de lectura.

Para transmitir estos datos Kafka crea Topics, flujos de datos que a su vez se dividen en particiones, cada mensaje se almacena en una de estas particiones.

\subsection{APACHE SAMZA}


\subsection{APACHE FLUME}


\subsection{Apache Spark}


\section{Visualización y análisis de datos}


\section{Motor de búsquedas}

\subsection{Algolia}


\subsection{ElectricSearch}


\subsection{Solr}