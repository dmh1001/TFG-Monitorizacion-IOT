\capitulo{2}{Objetivos del proyecto}

El objetivo de este proyecto consta de la captación dinámica de diversos datos provenientes de sensores IOT. Los datos, cedidos por el cliente, se han de tratar y almacenar en una base de datos para poder ser consultados y monitorizados por el usuario en cualquier momento. 

Se implementará también un algoritmo de aprendizaje automático que ha de aprender con los datos obtenidos a identificar patrones que puedan llevar a situaciones no deseadas y que esto pueda ayudar al usuario a tomar decisiones con la suficiente antelación.

\subsection{Objetivos}

\begin{enumerate}
    \item La instalación y configuración de un sistema capaz de recoger, almacenar y gestionar datos de sensores.
    \item La implementación de motores de búsqueda para facilitar encontrar los datos deseados con la mayor precisión posible.
    \item Que la monitorización de los datos se muestre lo más clara y accesible posible para el usuario.
    \item Realizar un algoritmo de aprendizaje automático que pueda predecir comportamientos y patrones en los datos de los sensores.
\end{enumerate}


\subsection{Objetivos personales}

\begin{enumerate}
    \item Utilizar un repositorio en github y aplicar conocimientos sobre gestión de proyectos.
    \item Adentrarme y aprender sobre tratamiento, procesado y gestión de datos.
    \item Tratar con dispositivos IoT.
\end{enumerate}
