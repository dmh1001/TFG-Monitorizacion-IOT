\apendice{Plan de Proyecto Software}

\section{Introducción}

\section{Planificación temporal}

Para la realización el proyecto se ha optado por la utilización el método \textbf{Scrum}, una metodología ágil de gestión de proyectos en la  que se ha dividido el desarrollo en \textbf{sprints} de 2 semanas de duración en la que había una reunión con el tutor y se planteaban las próximas tareas que se iban a realizar.

Debido a la situación actual del COVID-19 las reuniones se llevaron a cabo mediante videallamadas por \textbf{Microsoft Teams}

\subsection{Sprint 0 - Introducción (30/11/2020 - 14/12/2020 )}

Durante este \textbf{Sprint} lo principal fue investigar sobre las posibles herramientas que podía utilizar para este proyecto así cómo documentar dichas herramientas en la memoria.

Durante esta parte del proyecto se creo el repositorio de Github.

\subsection{Sprint 1   (14/12/2020 - 04/01/2021 )}

Durante este \textbf{Sprint} se decidieron las herramientas que se iban a utilizar así cómo redactar los objetivos del proyecto.

\subsection{Sprint 2   (04/01/2021 - 06/02/2021 )}

Durante este \textbf{Sprint} se instaló el servidor PRTG en una máquina virtual Windows 10 así cómo ElasticSearch. 

\subsection{Sprint 3   (06/02/2021 - 25/02/2021 )}

Durante este \textbf{Sprint} se realizo 
\subsection{Sprint 4   (25/02/2021 - 12/03/2021 )}
Durante este \textbf{Sprint} se migro ElasticSearch a una máquina Virtual Ubuntu server y se realizo la conexión entre ambas.

\subsection{Sprint 5   (12/03/2021 - 26/03/2021 )}



\subsection{Sprint 6   (12/03/2021 - 12/04/2021 )}


\section{Estudio de viabilidad}

\subsection{Viabilidad económica}

\subsection{Viabilidad legal}


